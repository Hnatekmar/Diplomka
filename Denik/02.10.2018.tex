\documentclass[a4paper,11pt,twoside]{article}
\usepackage{geometry}
\usepackage{advdate}
\geometry{a4paper, top=1.5cm, left=2.5cm, right=3.0cm, bottom=2.0cm, includehead, includefoot}
\usepackage[czech]{babel}
\usepackage[utf8]{inputenc}
\usepackage{graphicx}
\usepackage[pdf]{graphviz}
\newcommand{\tomorrow}{{\AdvanceDate[1]\today}}
\title{Konzultace}
\date{\today}
\begin{document}
	\pagenumbering{gobble}
	\maketitle
	Témata k diskuzi
	\begin{enumerate}
		\item Jaká by měla být reakční doba vozidla?
		\item Probrat ovládací schéma
			\begin{enumerate}
				\item Senzory
				\item Ovládání neuronovou sítí
			\end{enumerate}
		\item Simulační prostředí
			\begin{enumerate}
				\item Statické vs dynamické
				\item Délka evaluace
				\item Rozšíření
			\end{enumerate}
		\item Javascript
			\begin{enumerate}
				\item ES6 import v knihovnách
			\end{enumerate}
	\end{enumerate}
	
	\newpage
	\section{Poznámky}

	\newpage
	\section{Architektura aplikace}
	\begin{figure}[h!]
		\centering
		\digraph{architektura}{
			rankdir=TD;
			node [shape=box]
			simulace -> frontend;
			simulace -> backend;
		}
		\caption{architektura}
	\end{figure}
	Aplikace se dá rozdělit do tří částí a to: 
	\begin{enumerate}
		\item Simulace - Kompletní kód simulátoru, který slouží k vyhodnocování jednotlivých genomů
		\item Frontend - Webová aplikace pro zobrazování a simulaci
		\item Backend - Pro velmi rychlou evaluaci na serveru
	\end{enumerate}

	\subsubsection{Pravidla simulace}
	Simulace probíhá maximálně jednu minutu a končí v těchto případech:
	\begin{enumerate}
		\item Pokud se auto zastaví (má nulovou rychlost)
		\item V případě nárazu (agent je penalizován stržením konstantní hodnoty od jeho fitness skóre)
		\item Uplynutím časového limitu
    \end{enumerate}

	\subsection{Ovládání agenta}
	Agent je ovládán s pomocí níže uvedené neuronové sítě

	Kde neuronová síť volí mezi pohybem, což může být:
	\begin{enumerate}
		\item Dopredu plnou rychlostí vpřed
		\item Dozadu 15\% normální rychlosti
		\item Brzda zastavení / zpomalení
	\end{enumerate}
	a směrem, což může být:
	\begin{enumerate}
		\item Natočení volantu o 10 stupňů doleva
		\item Natočení volantu o 10 stupňů doprava
	\end{enumerate}
	\begin{figure}[!h]
		\centering
		\digraph{neural}{
			rankdir=LR;
			Senzory -> NN;
			Natoceni -> NN;
			NN -> Dopredu;
			NN -> Dozadu;
			NN -> Brzda;
			NN -> Doprava;
			NN -> Doleva;
		}
		\caption{Neuronová síť}
	\end{figure}
	\section{Backend}
	Backendová část běží s pomocí nodejs a pokouší se proces evaluace značně zrychlit rozložením simulace jednotlivých jedinců mezi fyzická jádra procesoru. Toto rozložení probíhá rovnoměrně a může tedy docházet k tomu, že jedno jádro čeká na zbytek (jelikož doba vyhodnocování je u jednotlivců různá).
	\section{Další práce}
	\begin{enumerate}
		\item Lepší rozložení jednotlivců u backendu (fronta?)
		\item Zrychlení a validace
		\item Rozšíření simulačního prostředí dle dohody
	\end{enumerate}
	\begin{figure}[!b]
		\flushright
		Ing. Jiří Lýsek, Ph.D.\hspace{0.5cm} \makebox[1.5in]{\hrulefill}
	\end{figure}
\end{document}
