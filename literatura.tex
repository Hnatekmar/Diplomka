\citace{beran}{Beran, 1994}{\autor{Beran, V.} \nazev{Typografický manuál.}
	Náchod: Nakladatelství Manuál, 1994. \\ ISBN 80-901824-0-2}

\citace{csn010166}{ČSN 01\,0166, 1992}{\nazev{ČSN 01\,0166 Nakladatelská
		(vydavatelská) úprava knih a~některých daląích druhů neperiodických
		publikací.} Praha: Federální úřad pro normalizaci a~měření, 1992}
\citace{csn016910}{ČSN 01\,6910, 1997}{\nazev{ČSN 01\,6910 Úprava písemností
		zpracovaných textovými editory nebo psaných strojem.} Praha: Český
	normalizační institut, 2004}
\citace{csniso31-0}{ČSN ISO 31-0, 1994}{\nazev{ČSN ISO 31-0 -- Veličiny
		a~jednotky. Část 0: Vąeobecné zásady}. Praha: Český normalizační institut,
	1994}

\citace{csniso31-11}{ČSN ISO 31-11, 1999}{\nazev{ČSN ISO 31-0 -- Veličiny
		a~jednotky. Část 11: Matematické znaky a~značky pouľívané ve fyzikálních
		vědách a~v~technice}. Praha: Český normalizační institut, 1999}

\citace{csniso690}{ČSN ISO 690, 1996}{\nazev{ČSN ISO 690 -- Bibliografické
		citace. Obsah, forma a~struktura}. Praha: Český normalizační institut, 1996}

\citace{csniso690-2}{ČSN ISO 690\spoj{}2, 2000}{\nazev{ČSN ISO 690-2 --
		Informace a~dokumentace -- Bibliografické citace -- Část 2: Elektronické
		dokumenty a~jejich části}. Praha: Český normalizační institut, 2000}

\citace{csn7144}{ČSN ISO 7144, 1996}{\nazev{ČSN ISO 7144 Dokumentace --
		Formální úprava disertací a~podobných dokumentů.} Praha: Český normalizační
	institut, 1996}
\citace{Nohel}{Nohel, 1972}{\autor{Nohel, F.} \nazev{Sazba matematická a
		chemická}. Praha: SNTL, 1972}
\citace{on882503}{ON 88\,2503, 1974}{\nazev{ON 88\,2503 Základní pravidla
		sazby.} Praha: Vydavatelství Úřadu pro normalizaci a~měření, 1974}
\citace{pop}{Pop, Fléger a~Pop, 1989}{\autor{Pop, P., Flégr, J., Pop, V.}
	\nazev{Sazba I~-- Ruční sazba.} Praha: SPN, 1989}
\citace{latzac}{Rybička, 2003}{\autor{Rybička, J.} \nazev{\LaTeX{} pro
		začátečníky.} 3. vyd. Brno: Konvoj, 2003}
\citace{TalDipl}{Talandová, 2006}{\autor{Talandová, P.} \nazev{Přístupy ve
		zpracování tabulek v~systémech DTP}. Diplomová práce. Brno: PEF MZLU v~Brně,
	2006}
@BOOK{Češka1994,
	title = {Petriho {S}ítě},
	publisher = {Akademické nakladatelství CERM Brno},
	year = {1994},
	author = {Milan ČEŠKA},
	abstract = {Učebnice Petriho sítí. Obsahuje definici PS, rozebírá
		základní problémy analýzy PS, invarianty, jazyky PS. Nezabývá se
		stochastickými, barvenými PS, ani dalšími rozšířeními. Končí
		u PT Petriho sítí s inhibitory.},
	isbn = {80-85867-35-4},
	owner = {David Martinek},
}
