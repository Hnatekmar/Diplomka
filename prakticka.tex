\kapitola{Vlastní práce}
Vlastní práce se skládá ze dvou částí klientská část, která slouží k vizualizaci algoritmu a zobrazení výsledků ze serverové části. Serverová část pro maximální zrychlení simulace. 
\sekce{Simulace}
\podsekce{Fitness funkce}
Fitness funkce je důležitou součástí simulace, která zásadně ovlivňuje chování výsledných agentů.
\sekce{Serverová část}
Serverová část vyhodnocuje jednotlivé jedince distribuovaně s pomocí fronty úkolů. Frontu poskytuje knihovna \textbf{bull}, která používá \textbf{redis} pro správu údajů o jednotlivých úkolech.

\podsekce{Výpočetní cluster}
Ukázalo, že vyhodnocování simulace zabírá neúměrné množství času a to i na nejvýkonnějším dostupném počítači. 
Například vyhodnocení jedné generace populace o 1024 jedincích zabralo ~290 s na nejsilnějším dostupném pc. Z tohoto důvodu bylo rozhodnuto o distribuce výpočetní zátěže mezi více počítačů. Byl vytvořen výpočetní cluster s následující specifikací:

\begin{tabular}{|l|c|c|c|}
	\hline 
	Procesor & RAM & Počet & Architektura\\ 
	\hline 
	S5P6818 Octa core & 1 GB & 2 & arm64 \\ 
	\hline 
	Broadcom BCM2837B0 quad-core & 1 GB & 1 & arm32 \\ 
	\hline 
	Phenom X4 965 & 8 GB & 1 & x64 \\ 
	\hline
	Intel Core i5-2300 & 4 GB & 1 & x64 \\ 
	\hline
	AMD A4-4300M & 4 GB & 1 & x64 \\ 
	\hline 
	Intel atom x5-Z8350 & 2 GB & 1 & x64 \\ 
	\hline
\end{tabular} 

Pro snadnou distribuci a správu byly všechny počítače zorganizovány do docker swarmu. Docker swarm obsahoval jednoho managera, který zároveň spouštěl klientskou aplikaci a další služby:

\begin{enumerate}
	\item \textbf{Portainer} pro správu clusteru
	\item \textbf{Arena} pro správu \textbf{bull}
	\item \textbf{redis} používaný knihovnou \textbf{bull}
\end{enumerate}

\podsekce{Průběh vyhodnocování}
Serverová část pracuje dle diagramu \ref{Diagram}, kde je vidět, že klient zadává do fronty úkoly (genom a nastavení simulace) a jednotliví zpracovatelé (počítače v clusteru), kteří si je z ní vyberou, jednotlivé genomy vyhodnotí a hodnotu fitness funkce pošlou zpět na klienta, který jakmile dostane všechny hodnoty zpět provede genetický algoritmu (mutace, křížení, ...) a novou generaci pošle znovu na vyhodnocení.
Tento přístup má několik výhod a to:
\begin{enumerate}
	\item Robustnost - Pokud jeden nebo více zpracovatelů selže (je například odpojen ze sítě) je možné pokračovat ve vyhodnocování (neúspěšný úkol se automaticky vrátí zpátky do fronty)
	\item Dobré rozložení zátěže - Jelikož si zpracovatel vytahuje úkoly z fronty je vždy optimálně zatížen a není třeba řešit rozložení mezi různě výkonnými a zatíženými počítači.
	\item Možnost vyhodnocení více úkolů zároveň - Jelikož se vyhodnocují jednotlivé genomy lze spustit i více simulací zároveň bez většího dopadu na výkon výpočtů.
\end{enumerate}

Lze i namítnout, že se zde projevuje určitá režie při síťové komunikaci se serverem, což může být zdrojem určitého zpomalení. Nicméně se toto zpomalení neprojevilo v průběhu testování clusteru.


\sekce{Klientská část}
\sekce{Simulace}
\podsekce{Agent}
Definice samotného agenta zásadně ovlivňuje výsledek simulace, protože stanovuje vstupy a výstupy do a z neuronové sítě. 
Samotný agent je auto, které je vybaveno několika vzdálenostními senzory (v závislosti na konfiguraci viz. níže), které mu poskytují 
\sekce{Experimenty}
\kapitola{Závěr a zhodnocení} 

https://snapshot.raintank.io/dashboard/snapshot/FmoHRo3sfdk8eAQlkRc4rfKK3YF043oE?orgId=2