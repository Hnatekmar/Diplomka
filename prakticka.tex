\kapitola{Vlastní práce}
Vlastní práce se skládá ze dvou částí klientská část, která slouží k vizualizaci algoritmu a zobrazení výsledků ze serverové části. Serverová část pro maximální zrychlení simulace. 
\sekce{Simulace}
\podsekce{Fitness funkce}
Fitness funkce je důležitou součástí simulace, která zásadně ovlivňuje chování výsledných agentů.
\sekce{Serverová část}
Serverová část obaluje simulaci do jednoduchého REST api. V rámci co největší rychlosti byl vytvořen i jednoduchý klient, který provádí algoritmus NEAT a pro vyhodnocení jednotlivých jedinců využívá toto API. API má jen jeden endpoint a to \bf{/evaluate}, který přebírá pole json objektů z nichž každý obsahuje informace o neuronové síti vygenerované s pomocí knihovny NEAT.
Ačkoliv se může zdát, že tento přístup bude pomalejší díky přídavné režii, kterou s sebou nese síťová komunikace tento přístup má i své výhody. Hlavní výhodou je možnost rozložení samotných výpočtů mezi několik počítačů.
\sekce{Klientská část}
\sekce{Simulace}