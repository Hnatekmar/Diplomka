\kapitola{Úvod a cíl práce}
\sekce{Úvod do problematiky}
S růstem výpočetního výkonu a rozvojem \textbf{gpugpu} (paralelizace výpočtů na grafické kartě) se neuronové sítě ukázaly jako mocný nástroj pro řešení složitých problémů na které standardní metody umělé inteligence nestačily. Především díky jejích schopn


\kapitola{Teorie}
\sekce{Neuron}
$$\sigma(\Sigma_{i=0}^{N} \theta \cdot x_{i} + b)$$
\sekce{Aktivační funkce}
Aktivační funkce se používá pro definování výstupu a zavedení nelinearity. 

\sekce{Linearní funkce}
$$f(x) = x$$
\sekce{Sigmoid}
$$f(x) = \frac{1}{1 + e^{-x}}$$
\sekce{RELU}
\[ 
f(x) = 
\begin{dcases*} 
\text{$x>=0$,} & $x$ \\ 
\text{$x<0$,} & 0 
\end{dcases*} 
\]