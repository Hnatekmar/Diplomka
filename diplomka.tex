% preambule dokumentu
\documentclass[12pt]{article}
\usepackage[utf8]{inputenc}
\usepackage[czech]{babel}
\usepackage{cite}
\usepackage[pdf]{graphviz}
\usepackage{mathtools}
\usepackage{dipp8, xltxtra, graphicx}
\usepackage{pgfplots}

\begin{document}
\pagestyle{headings}

\def\typprace{Diplomová}

% uvodni cast zaverecne prace
\titul{Řízení autonomního agenta pomocí neuroevoluce}{Bc. Martin Hnátek}{Ing. Jiří Lýsek, Ph.D.}{Brno, 2018}
\podekovani{Rád bych zde poděkoval svému vedoucímu Ing. Jiří Lýskovi, Ph.D. za jeho rady a čas, který mi věnoval při řešení této práce.}
\prohlasenimuz{V Brně dne \today}

\abstract{Autonomous agent control using neuroevolution}{}
% TODO: Lepší abstract
\abstrakt{Řízení autonomního agenta pomocí neuroevoluce}{Tato práce se zabývá trénováním autonomního agenta - auta s pomocí algoritmu neuroevoluce. Toto zahrnuje tvorbu simulačního prostředí pro agenta, vhodným návrhem agenta (senzorů a řízení) a také návrhem fitness funkce}
\obsah
\cislovat{2}

% jednotlive kapitoly dokumentu
\kapitola{Úvod a cíl práce}
\sekce{Úvod do problematiky}
S růstem výpočetního výkonu a rozvojem \textbf{gpugpu} (paralelizace výpočtů na grafické kartě) se neuronové sítě ukázaly jako mocný nástroj pro řešení složitých problémů na které standardní metody umělé inteligence nestačily.\\
Další oblastí umělé inteligence, které nárust masivní paralelizace prospěl jsou metaheuristiky, které mohou  s nárůstem výpočetního výkonu v rozmném čase pokrýt stále větší stavový prostor a jsou tedy schopné rychle řešit stále složitější problémy.\\
Neuroevoluce propojuje oba přístupy a využívá je ke generování topologii neuronových sítí a nastavení jejích vah. Výsledkem je neuronová síť, jejiž architektura lépe popisuje daný problém a lze jí aplikovat i na problémy na které by klasické neuronové sítě aplikovat nešly. Jedná se například o problémy u kterých je těžké získat trénovací data a nelze tedy neuronovou síť natrénovat klasickými metodami jako je backpropagace.\\

\sekce{Cíl práce}
Cílem této práce je navrhnout a pokusit se natrénovat autonomního agenta v simulovaném prostředí s pomocí algoritmu neuroevoluce.

\kapitola{Neuronové sítě}

Neuronové sítě jsou model strojového učení, který je volně založený na principu zvířecího mozku.  \cite[s.~41]{fundementalsOfDeepLearning}

\sekce{Neuron}
Neuron je základní  výpočetní jednotka neuronových sítí, která je definovaná jako suma všech jejích vstupů a aplikace aktivační funkce.
	$$\sigma(\Sigma_{i=0}^{N} \theta_i \cdot x_{i} + b)$$
Kde:
\begin{enumerate}
	\item $\sigma$ - aktivační funkce
	\item $\theta$ - skrytá váha pro daný vstup
	\item b - bias neuronu
\end{enumerate}

\sekce{Vrstva}
Vrstva je skupina neuronů se stejnou aktivační funkcí. 

\podsekce{Aktivační funkce}
Aktivační funkce se používá pro definování výstupu a zavedení nelinearity. Bez nich by byla neuronová síť schopna aproximovat pouze n-dimenzionální rovinu. \cite[s.~65]{fundementalsOfDeepLearning} \\
Dalším využitím je omezení výstupních hodnot. Například aktivační funkce sigmoid se s oblibou používá u výstupní vrstvy neuronových sítí určených ke klasifikačním problémům, protože je to relace $\mathbb{R} \rightarrow \{0..1\}$, která se dá jednoduše jako "jistota" neuronu, že se jedná o výstup, který neuron reprezentuje. Podobně se dá uvažovat i o funkcích jako je například softmax a tanh, které také najdou hojné využití u klasifikačních problémů.

\podsekce{Linearní funkce}
Vrací vstup, tak jak je. Využití najde především u vstupní vrstvy neuronové sítě a u neuronových sítí, které řeší regresní typy úloh.
\\
\begin{tikzpicture}
\begin{axis}[grid=major, xlabel=$x$, ylabel={$f(x)$}]
\addplot[blue, samples=100, smooth, unbounded coords=discard]
plot (\x, \x);
\end{axis}
\end{tikzpicture}
$$f(x) = x$$
\podsekce{Sigmoid}
Sigmoid je aktivační funkce, která je schopná potlačit extrémní hodnoty 

\begin{tikzpicture}
\begin{axis}[grid=major, xlabel=$x$, ylabel={$f(x)$}]
\addplot[blue, samples=100, smooth, unbounded coords=discard]{1 / (1 + e ^ (-\x))};
\end{axis}
\end{tikzpicture}
$$f(x) = \frac{1}{1 + e^{-x}}$$
\podsekce{Tanh}

Tanh je funkce obdobná sigmoidu. Hlavní rozdíl mezi ní a sigmoidem je ten, že její obor je v rozmezí -1 a 1 hodí se proto i pro záporná výstupy, které vyžadují záporná čísla. \cite[s.~67]{fundementalsOfDeepLearning}

\begin{tikzpicture}
\begin{axis}[grid=major, xlabel=$x$, ylabel={$f(x)$}]
\addplot[blue, samples=100, smooth, unbounded coords=discard]{tanh(x)};
\end{axis}
\end{tikzpicture}
$$f(x) = tanh(x)$$
\podsekce{RELU}
RELU je aktivační funkce, která je podobná lineární aktivační funkci s tím rozdílem, že pokud vstupní hodnota nepřesáhne určitého prahu výstupem je 0. Její hlavní výhodou je to, že zabraňuje problémům s takzvaným explodujícím gradientem \cite[s.~69]{fundementalsOfDeepLearning} \\

\begin{tikzpicture}
\begin{axis}[grid=major, xlabel=$x$, ylabel={$f(x)$}]
\addplot[blue, samples=100, smooth, unbounded coords=discard]{max(0, x)};
\end{axis}
\end{tikzpicture}
\[ 
f(x) = 
\begin{dcases*} 
\text{$x>=0$,} & $x$ \\ 
\text{$x<0$,} & 0 
\end{dcases*} 
\]
\sekce{Genetické algoritmy}
Genetické algoritmy slouží k řízenému prohledávání stavového prostoru založené na teorii evoluce. 
\podsekce{Princip}
Základní myšlenka spočívá ve vygenerování náhodných jedinců (řešení problému) a jejích postupné zlepšování s pomocí operací křížení a mutace. Proces zlepšování genomů probíhá na základě jeho hodnocení (fitness).

Samotný algoritmus pak lze rozdělit na následující kroky \cite[s.~12]{geneticAlgorithms}

\begin{enumerate}
	\item Vygeneruj náhodnou populaci o n chromozomech
	\item Pro každý chromozom spočítej jeho fitness
	\item Opakuj n krát
	\begin{enumerate}
		\item Vyber pár chromozomů na základě jejích ohodnocení (selekce)
		\item S určitou pravděpodobností $p_c$ rozděl pár chromozomů na náhodném místě a jejích spojením.
		\item S určitou pravděpodobností mutuj daného jedince
	\end{enumerate}
	\item Nahraď současnou populaci populací, která vznikla předchozím krokem
	\item Jdi na krok 2
\end{enumerate}

\podsekce{Kódování}
Způsob zápisu řešení problému. Existuje mnoho různých kódování a každý má své výhody a nevýhody. Zde je seznam několika nejpoužívanějších kódování \cite[s.~42-43]{geneticCZ}:
\begin{enumerate}
	\item Binární - řetězec bitů, který může například reprezentovat jednu nebo více numerických hodnot. 
	\item Realná čísla - Jedno nebo více reálných čísel
	\item Kombinatorické - Může být například seznam čísel označujících 
\end{enumerate}
\podsekce{Křížení}
Křížení je operátor, který kombinuje dva jedince do jednoho. Jeho implementace je samozřejmě závislá na 
\podsekce{Mutace}
Mutace náhodně modifikuje chromozom a zavádí tak do populace variaci.
\podsekce{Selekce}
Selekce je proces výběru dvou jedinců na něž jsou později aplikovány genetické operátory jako je křížení a mutace.  
\sekce{Neuroevoluce}

%\input{prvnicast.tex}
%\input{druhacast.tex}
%\input{zaverprace.tex}
\begin{literatura}
	\citace{fundementalsOfDeepLearning}{BUDUMA, Nikhil 2017}{
	\autor{BUDUMA, Nikhil.} \nazev{Fundamentals of deep learning: designing next-generation machine intelligence algorithms.} Sebastopol: O'Reilly, 2017. ISBN 978-149-1925-614.}
\end{literatura}
\prilohy{}
\priloha{CD se zdrojovým kódem}
\end{document}

