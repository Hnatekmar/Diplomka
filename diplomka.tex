% preambule dokumentu
\documentclass[12pt]{article}
\usepackage[utf8]{inputenc}
\usepackage[czech]{babel}
\usepackage{cite}
\usepackage[pdf]{graphviz}
\usepackage{mathtools}
\usepackage{dipp8, xltxtra, graphicx}
\usepackage{pgfplots}
\usepackage{amsfonts}
\usepackage{neuralnetwork}
\usepackage[hidelinks]{hyperref}
\usepackage{listings}
\usepackage{plantuml}

%% Definice javascriptu
\lstdefinelanguage{JavaScript}{%
	keywords={const, let, typeof, instanceof, new, true, false, catch, function, return, null, undefined, catch, switch, var, if, in, while, for, do, else, case, break},
	keywordstyle=\bfseries,
	ndkeywords={class, export, throw, import, this},
	ndkeywordstyle=\bfseries,
	sensitive=false,
	comment=[l]{//},
	morecomment=[s]{/*}{*/},
	commentstyle=\ttfamily,
	stringstyle=\color{blue}\ttfamily,
	morestring=[b]',
	morestring=[b]`,
	morestring=[b]"
}

\begin{document}
\pagestyle{headings}

\def\typprace{Diplomová}

% uvodni cast zaverecne prace
\titul{Řízení autonomního agenta pomocí neuroevoluce}{Bc. Martin Hnátek}{Ing. Jiří Lýsek, Ph.D.}{Brno, 2018}
\podekovani{Rád bych zde poděkoval svému vedoucímu Ing. Jiří Lýskovi, Ph.D. za jeho cenné rady a čas, který mi věnoval při řešení této práce.}
\prohlasenimuz{V~Brně dne \today}

\abstract{Autonomous agent control using neuroevolution}{}
% TODO: Lepší abstract
\abstrakt{Řízení autonomního agenta pomocí neuroevoluce}{Tato práce představuje teoretický základ neuroevoluce, tvorbou simulačního prostředí pro autonomního agenta a jeho natrénování s pomocí knihovny Neataptic.}
\obsah
\listoffigures
\newpage{}
\listoftables
\cislovat{2}

% jednotlive kapitoly dokumentu
\kapitola{Úvod a cíl práce}
\sekce{Úvod do problematiky}
S růstem výpočetního výkonu a rozvojem \textbf{gpugpu} (paralelizace výpočtů na grafické kartě) se neuronové sítě ukázaly jako mocný nástroj pro řešení složitých problémů na které standardní metody umělé inteligence nestačily.\\
Další oblastí umělé inteligence, které nárust masivní paralelizace prospěl jsou metaheuristiky, které mohou  s nárůstem výpočetního výkonu v rozmném čase pokrýt stále větší stavový prostor a jsou tedy schopné rychle řešit stále složitější problémy.\\
Neuroevoluce propojuje oba přístupy a využívá je ke generování topologii neuronových sítí a nastavení jejích vah. Výsledkem je neuronová síť, jejiž architektura lépe popisuje daný problém a lze jí aplikovat i na problémy na které by klasické neuronové sítě aplikovat nešly. Jedná se například o problémy u kterých je těžké získat trénovací data a nelze tedy neuronovou síť natrénovat klasickými metodami jako je backpropagace.\\

\sekce{Cíl práce}
Cílem této práce je navrhnout a pokusit se natrénovat autonomního agenta v simulovaném prostředí s pomocí algoritmu neuroevoluce.

\kapitola{Neuronové sítě}

Neuronové sítě jsou model strojového učení, který je volně založený na principu zvířecího mozku.  \cite[s.~41]{fundementalsOfDeepLearning}

\sekce{Neuron}
Neuron je základní  výpočetní jednotka neuronových sítí, která je definovaná jako suma všech jejích vstupů a aplikace aktivační funkce.
	$$\sigma(\Sigma_{i=0}^{N} \theta_i \cdot x_{i} + b)$$
Kde:
\begin{enumerate}
	\item $\sigma$ - aktivační funkce
	\item $\theta$ - skrytá váha pro daný vstup
	\item b - bias neuronu
\end{enumerate}

\sekce{Vrstva}
Vrstva je skupina neuronů se stejnou aktivační funkcí. 

\podsekce{Aktivační funkce}
Aktivační funkce se používá pro definování výstupu a zavedení nelinearity. Bez nich by byla neuronová síť schopna aproximovat pouze n-dimenzionální rovinu. \cite[s.~65]{fundementalsOfDeepLearning} \\
Dalším využitím je omezení výstupních hodnot. Například aktivační funkce sigmoid se s oblibou používá u výstupní vrstvy neuronových sítí určených ke klasifikačním problémům, protože je to relace $\mathbb{R} \rightarrow \{0..1\}$, která se dá jednoduše jako "jistota" neuronu, že se jedná o výstup, který neuron reprezentuje. Podobně se dá uvažovat i o funkcích jako je například softmax a tanh, které také najdou hojné využití u klasifikačních problémů.

\podsekce{Linearní funkce}
Vrací vstup, tak jak je. Využití najde především u vstupní vrstvy neuronové sítě a u neuronových sítí, které řeší regresní typy úloh.
\\
\begin{tikzpicture}
\begin{axis}[grid=major, xlabel=$x$, ylabel={$f(x)$}]
\addplot[blue, samples=100, smooth, unbounded coords=discard]
plot (\x, \x);
\end{axis}
\end{tikzpicture}
$$f(x) = x$$
\podsekce{Sigmoid}
Sigmoid je aktivační funkce, která je schopná potlačit extrémní hodnoty 

\begin{tikzpicture}
\begin{axis}[grid=major, xlabel=$x$, ylabel={$f(x)$}]
\addplot[blue, samples=100, smooth, unbounded coords=discard]{1 / (1 + e ^ (-\x))};
\end{axis}
\end{tikzpicture}
$$f(x) = \frac{1}{1 + e^{-x}}$$
\podsekce{Tanh}

Tanh je funkce obdobná sigmoidu. Hlavní rozdíl mezi ní a sigmoidem je ten, že její obor je v rozmezí -1 a 1 hodí se proto i pro záporná výstupy, které vyžadují záporná čísla. \cite[s.~67]{fundementalsOfDeepLearning}

\begin{tikzpicture}
\begin{axis}[grid=major, xlabel=$x$, ylabel={$f(x)$}]
\addplot[blue, samples=100, smooth, unbounded coords=discard]{tanh(x)};
\end{axis}
\end{tikzpicture}
$$f(x) = tanh(x)$$
\podsekce{RELU}
RELU je aktivační funkce, která je podobná lineární aktivační funkci s tím rozdílem, že pokud vstupní hodnota nepřesáhne určitého prahu výstupem je 0. Její hlavní výhodou je to, že zabraňuje problémům s takzvaným explodujícím gradientem \cite[s.~69]{fundementalsOfDeepLearning} \\

\begin{tikzpicture}
\begin{axis}[grid=major, xlabel=$x$, ylabel={$f(x)$}]
\addplot[blue, samples=100, smooth, unbounded coords=discard]{max(0, x)};
\end{axis}
\end{tikzpicture}
\[ 
f(x) = 
\begin{dcases*} 
\text{$x>=0$,} & $x$ \\ 
\text{$x<0$,} & 0 
\end{dcases*} 
\]
\sekce{Genetické algoritmy}
Genetické algoritmy slouží k řízenému prohledávání stavového prostoru založené na teorii evoluce. 
\podsekce{Princip}
Základní myšlenka spočívá ve vygenerování náhodných jedinců (řešení problému) a jejích postupné zlepšování s pomocí operací křížení a mutace. Proces zlepšování genomů probíhá na základě jeho hodnocení (fitness).

Samotný algoritmus pak lze rozdělit na následující kroky \cite[s.~12]{geneticAlgorithms}

\begin{enumerate}
	\item Vygeneruj náhodnou populaci o n chromozomech
	\item Pro každý chromozom spočítej jeho fitness
	\item Opakuj n krát
	\begin{enumerate}
		\item Vyber pár chromozomů na základě jejích ohodnocení (selekce)
		\item S určitou pravděpodobností $p_c$ rozděl pár chromozomů na náhodném místě a jejích spojením.
		\item S určitou pravděpodobností mutuj daného jedince
	\end{enumerate}
	\item Nahraď současnou populaci populací, která vznikla předchozím krokem
	\item Jdi na krok 2
\end{enumerate}

\podsekce{Kódování}
Způsob zápisu řešení problému. Existuje mnoho různých kódování a každý má své výhody a nevýhody. Zde je seznam několika nejpoužívanějších kódování \cite[s.~42-43]{geneticCZ}:
\begin{enumerate}
	\item Binární - řetězec bitů, který může například reprezentovat jednu nebo více numerických hodnot. 
	\item Realná čísla - Jedno nebo více reálných čísel
	\item Kombinatorické - Může být například seznam čísel označujících 
\end{enumerate}
\podsekce{Křížení}
Křížení je operátor, který kombinuje dva jedince do jednoho. Jeho implementace je samozřejmě závislá na 
\podsekce{Mutace}
Mutace náhodně modifikuje chromozom a zavádí tak do populace variaci.
\podsekce{Selekce}
Selekce je proces výběru dvou jedinců na něž jsou později aplikovány genetické operátory jako je křížení a mutace.  
\sekce{Neuroevoluce}

\kapitola{Metodika}
Na začátku je potřeba provést podrobnou analýzu problému s ohledem na požadavky stanovené konzultacemi s vedoucím. Návrh bude také zahrnovat experimenty, které budou s neuroevolucí provedeny. \\
Po návrhu bude následovat implementace daného řešení a jehož popis bude přidán do této práce. V průběhu implementace bude také kód a návrh postupně upravován na základě požadavků, které se mohou objevit až při implementaci navrženého řešení. \\
Po implementaci bude následovat realizace a vyhodnocení experimentů popsaných v návrhu. \\
Na závěr proběhne vyhodnocení řešení s návrhem možných zlepšení.
\kapitola{Analýza problému}
Tato kapitola se zabývá analýzou funkčních a nefunkčních požadavků pro softwarové řešení. Požadavky vznikaly na základě konzultace s vedoucím a vlastní invencí.

\sekce{Funkční požadavky}
Funkční požadavky jsou rozděleny do několika kategorii.
\podsekce{Vizualizace}
Výsledky bude třeba jak pro testování algoritmu, tak pro efektní zobrazení.
\podsekce{NEAT}
Použití algoritmu NEAT zahrnuje několik požadavků pro jeho správnou funkčnost.
\begin{itemize}
	\item Testovací prostředí musí být pro všechny agenty stejné
	\item Fitness funkce by měla být deterministická
	\item Program, který vyhodnocuje 
\end{itemize}

\sekce{Nefunkční požadavky} 
\begin{itemize}
	\item Rychlost - řešení musí být, co nejrychlejší
\end{itemize}


\kapitola{Návrh řešení}
Tato kapitola obsahuje návrh řešení založený na požadavcích identifikovaných v předchozí kapitole. Na základě těchto požadavků byl návrh rozdělen do dvou projektů a to webového frontendu, který slouží jako rozhraní ukazující průběh simulace v realném čase a serverové části, která slouží pro rychlý výpočet simulace bez vykreslovaní na platformě Node.js.

Jelikož obě části budou používat stejný simulační kód bylo rozhodnuto, že bude samotná simulace vytvořena jako softwarová knihovna. Tuto závislost ilustruje obrázek \ref{fig:dependency}.
\begin{figure}[h!]
	\centering
	\includegraphics[width=0.4\linewidth]{architektura}
	\caption{Schéma závislostí}
	\label{fig:dependency}
\end{figure}


\sekce{Simulace}
Z obrázku \ref{fig:dependency} je zřejmé, že simulační kód spojuje obě části dohromady. Je tedy důležité, aby byl navržený tak, aby jej bylo co nejjednodušeji integrovat s oběma řešeními.

Na základě těchto požadavků a podmínek, které jsou stanoveny v předchozí kapitole byl vytvořen následující návrh:

\label{sec:ECS}
Simulace obsahuje vlastní engine, který je navržen v duchu ECS (\textbf{Entity component system}), podrobný popis lze nalézt v sekci \ref{sec:ces}. Důvodů k implementaci vlastního enginu oproti použití existujících řešení je několik. Hlavním je nutnost separace grafické reprezentace od 

Není tedy žádným překvapením, že se všechny komponenty nalezené v simulaci dají rozložit na systémy, komponenty a entity. Pro lepší představu o implementaci je níže uveden přehled všech systému, entit a komponent použitých v simulaci.

\podsekce{Diagram tříd}
Níže popsaný návrh doplňuje diagram tříd, který zobrazuje jejích návaznost a zároveň také do návrhu zavádí třídu \textbf{Simulation}, která celou simulaci schovává za jednoduché rozhraní a umožňuje snadnou integraci simulace do různých aplikací.

\begin{figure}[H]
	\centering
 	\includegraphics[width=0.7\linewidth]{classDiagram}
	\caption{Diagram tříd}
	\label{fig:class-diagram}
\end{figure}

\podsekce{Entity}
Simulace obsahuje následující entity:

\textbf{PhysicsGroup} Seskupuje fyzikální entity do jedné pro snadnou manipulaci s nimi.

\textbf{RoadPart} Entita, která obaluje jednu nebo více překážek tak, aby se s nimi dalo snadno pohybovat. Používá se pro tvorbu složitějších dílů vozovky.

\textbf{Car} Reprezentuje samotné vozidlo, obsahuje jak jeho grafickou reprezentaci, tak kompletní logiku a fyzikální model.

\podsekce{Komponenty}
Simulace obsahuje následující komponenty:
\begin{itemize}
	\item \textbf{Car} obsahuje všechny potřebné informace o agentovi. Toto zahrnuje vše od neurnové sítě, která je použitá pro jeho řízení po ovládání jednotlivých kol agenta.
	\item \textbf{Graphics} komponenta, která obsahuje grafické informace pro \textbf{Pixi.js}.
	\item \textbf{Physics} komponent, která obsahuje fyzikální entity pro \textbf{P2.js}
\end{itemize}

\podsekce{Systémy}
Návrh simulace obsahuje následující systémy:

\textbf{Car} systém, který se stará o ovládání agenta a částečně o vyhodnocování jeho fitness. Pro každý snímek prožene vstupy (v závislosti na konfiguraci) neuronovou sítí a na základě jejích výstupů řídí agenta.

\textbf{Graphics} grafický systém, který slouží především k překreslování entit.

\textbf{Physics} krokuje fyzikální engine a synchronizuje grafickou reprezentaci s fyzikální entitou. Tento proces probíhá pro každý snímek a skládá se z přiřazení nové rotace a pozice pro grafickou entitu.

\textbf{RoadDirector} Road director se stará o generování nekonečného prostředí pro agenta. Děje se tak na základě předefinovaných dílu vozovky z nichž každý zaplňuje celou obrazovku simulace. V případě, že agent dorazí až na konec obrazovky, je mu určen nový navazující dílek. Agent je pak přesunut na opačnou stranu obrazovky a zároveň je vyměněn díl na kterém se nachází. Tento postup a důvod jeho návrhu je rozepsaný v sekci \ref{sec:simulationRoadDirector}.

\sekce{Klientská část}
Klientská část je webové rozhraní, které vzniklo z požadavků na vizualizaci a ověření funkčnosti simulační knihovny. 

Z požadavků vyšla aplikace, která obsahuje 2 rozdílné obrazovky a to první, která slouží pro zobrazení a simulaci vývoje algoritmu v reálném čase. Další slouží pro zpětné přehrávání již vygenerovaných genomů.

\podsekce{Zobrazení algoritmu v realném čase}
Zobrazení algoritmu v reálném čase nabízí jak nadhled na průběh algoritmu s pomocí grafu, který bude zobrazovat dosaženou fitness jednotlivých generací a přímým zobrazením průběhu jednotlivých jedinců (v náčrtu zobrazeno jako obdélníky).
\begin{figure}[H]
	\centering
	\includegraphics[width=0.6\linewidth]{wireframe/main}
	\caption{Drátkový návrh zobrazení algoritmu v reálném čase}
	\label{fig:main}
\end{figure}

\podsekce{Přehrávač}
Cílem přehrávače je zobrazit průběh již vyhodnoceného jedince. K tomuto má textové pole s informacemi o jedinci, tlačítko pro spouštění přehrávání a plocha na které je vizualizován průběh agenta.
\begin{figure}[H]
	\centering
	\includegraphics[width=0.6\linewidth]{wireframe/player}
	\caption{Drátkový návrh přehrávače}
	\label{fig:player}
\end{figure}


\sekce{Serverová část}
Serverová část byla nakonec navržena a vytvořena ve dvou na sebe navazujících verzích. 

\podsekce{První verze}
Je ilustrovaná na obrázku \ref{fig:server_first}. Návrh první verze popisuje jednoduchou aplikaci, která rozkládá vyhodnocování jednotlivých genomů mezi jednoho nebo více zpracovatelů. Každý zpracovatel běží ve vlastním vlákně a zátěž je tedy rozložena mezi dostupná jádra procesoru. Zpracovatel po vyhodnocení genomu vrací hlavnímu vláknu fitness daného jedince. Ten si ho uloží a po vyhodnocení všech jedinců tímto způsobem provede algoritmus NEAT. Tento proces se opakuje do té doby, než dojde k naplnění ukončujících podmínek (maximální počet generací) nebo přerušení programu.

\begin{figure}[H]
	\centering
	\includegraphics[width=0.7\linewidth]{server_first_use_case}
	\caption{První verze serverové části}
	\label{fig:server_first}
\end{figure}

\podsekce{Druhá verze}
Druhá verze rozšiřuje návrh o možnost rozložení výpočtů mezi více počítačů. Byla navržená poté, co bylo zjištěno, že předchozí verze nebyla schopná vyhodnotit dostatečné množství genomů dostatečně rychle.

Pracuje dle diagramu \ref{fig:distributed}, kde je vidět, že klient zadává do fronty úkoly (genom a nastavení simulace). Jednotliví zpracovatelé (počítače v~clusteru), si je z~ní vyberou, genomy a vyhodnotí a hodnotu fitness funkce pošlou zpět na klienta. Toto je ilustrováno v diagramu \ref{fig:serverusecase}. 

\begin{figure}[H]
	\centering
	\includegraphics[width=0.7\linewidth]{server_use_case}
	\caption{Sekvenční diagram komunikace se serverem}
	\label{fig:serverusecase}
\end{figure}

Jakmile klient dostane všechny hodnoty zpět provede na populaci algoritmus NEAT (mutace, křížení, \dots) a poté je nová generace poslána znovu na vyhodnocení.

\podsekce{Databáze}
Všechny verze serverové části zaznamenávají průběh algoritmu NEAT do databáze, jejíž schéma lze vidět v obrázku \ref{fig:database}. Databáze obsahuje tabulku \textbf{Genomes}, která uchová fitness nejlepšího a nejhoršího jedince v generaci, zároveň také obsahuje nejlepší genom z dané generace. Každý záznam v tabulace \textbf{Genomes} má přidělenou konfiguraci se kterou byl spuštěn. Toto je důležité při vyhodnocování více konfigurací zároveň, které je popsáno níže.

\begin{figure}[H]
	\centering
	\includegraphics[width=0.7\linewidth]{ERD}
	\caption{Schéma databáze}
	\label{fig:database}
\end{figure}

\sekce{Návrh experimentů a simulačního prostředí}
Experiment se bude skládat z agenta - vozidla, který bude vložen do simulovaného prostředí (sekce \ref{sec:simulationEnvironment}). Agent bude řízen neuronovou sítí (více v sekci \ref{sec:agent}).
\podsekce{Simulační prostředí}
\label{sec:simulationRoadDirector}
Návrh simulačního prostředí je nesmírně důležitý, protože definuje překážky, které musí agent překonat. Právě s pomocí simulačního prostředí lze postupně měnit obtížnost, jak je to stanoveno v zadání diplomové práce.

Po zvážení různých způsobů implementace simulačního prostředí bylo rozhodnuto, že se návrh vydá cestou procedurálního generátoru map. 
Ten bude fungovat tak, že se nejdříve nadefinují dílky prostředí ve kterých se bude agent pohybovat. Definice bude obsahovat dvě zásadní informace a o to prostředí, které dílek reprezentuje, což jsou informace o vozovce, dynamické prvky jako jsou třeba další auta semefory a jiné. Zároveň bude obsahovat i informaci o tom na které další dílky navazuje (například silnice ve tvaru I může ze shora navazovat na sebe a na odbočku ve doleva). Procedurální generátor pak agenta vloží do specifikovaného dílku a při překročení stanovených hranic dílku (například pokud agent vyjede z obrazovky), dílek náhodně zamění za jiný navazující a agenta přesune na patřičnou pozici. 

Tento přístup má následující výhody:
\begin{itemize}
	\item Nízká paměťová náročnost. Stačí si uchovávat jen definice dílků
	\item Možnost generování rozmanitých testovacích prostředí
	\item Nízké nároky na síťový přenos - Není třeba přenášet mapu při změně konfigurace
\end{itemize}

\podsekce{Agent}
\label{sec:agent}
Definice agenta zásadně ovlivňuje výsledek simulace, protože stanovuje vstupy a výstupy do a z~neuronové sítě. V této práci bude testováno několik konfigurací, jejíchž popis lze nalézt v této sekci a v sekci \ref{sec:extensions}.

V případě této práce je agentem auto, které je vybaveno vzdálenostními senzory. Měření těchto senzorů je normalizováno (maximální vzdálenost měřícího paprsku je 800 m) a předáno jako vstup do neuronové sítě.

Řízení agenta bude probíhat tak, že se každý snímek s~pomocí neuronové sítě na obrázku \ref{fig:control_network} rozhoduje, jakou akci podnikne. Má následující možnosti:

\begin{enumerate}
	\item Ovládání volantu 
	\begin{enumerate}
		\item $z_1$ Otočení volantem o určitý počet stupňů doleva
		\item $z_2$ Otočení volantem o určitý počet stupňů doprava
	\end{enumerate} 
	\item Rychlostní stupně
	\begin{enumerate}
		\item $z_3$ - zpátečka
		\item $z_4$ - rychlosti dopředu
	\end{enumerate}
\end{enumerate}

Ovládání volantu i volba rychlostního stupňů probíhá zároveň a to tak, že se vždy z dané skupiny neuronů vybere ten, který má největší hodnotu. Tento přístup je identický tomu, který se používá například u neuronových sítí pro klasifikaci.

\begin{figure}[H]
	\centering
	\includegraphics[width=0.7\linewidth]{AgentSchema}
	\caption{Řízení agenta}
	\label{fig:agentschema}
\end{figure}

%digraph { 
%	rankdir=LR
%	graph [ nodesep=0.5] 
%	node [shape=box] 
%	"Senzor" -> "Agent"
%	"Agent" -> "Akce"
%	"Akce" -> "Svet"
%	"Svet" -> "Senzor"
%}

\begin{figure}[H]
	\centering
	\begin{neuralnetwork}[height=7]
		\newcommand{\nodetextx}[2]{\ifthenelse{\equal{#2}{0}}{$b_0$}{$s_{#2}$}}
		\newcommand{\nodetextz}[2]{$z_#2$}
		\newcommand{\nodetexth}[2]{\ifthenelse{\equal{#2}{0}}{$b_1$}{$h_{#2}$}}
		\inputlayer[count=6, title={Data ze senzorů}, text=\nodetextx]
		\hiddenlayer[count=6, title={Skrytá vrstva}]
		\linklayers
		\outputlayer[count=5, title={Výstup}, text=\nodetextz] 
		\linklayers
	\end{neuralnetwork}
	\caption{Neuronová síť agenta}
	\label{fig:control_network}
\end{figure}

\podsekce{Možné konfigurace agenta} 
\label{sec:extensions}
Výše uvedenou konfiguraci lze rozšířit o níže uvedené vstupy/výstupy.

Modifikace vstupů umožní agentovi vnímat více než jen vzdálenosti z jednotlivých senzorů. Bude zajímavé pozorovat, jak se agent s jednotlivými vjemy poradí. V práci se vyzkouší následující přidané vstupy:
\begin{itemize}
	\item aktuální rychlost
	\item náklon volantu
\end{itemize}

Modifikovat lze také výstupy neuronové sítě toto rozšíří nebo omezí možnosti agenta:
\begin{itemize}
	\item Přidáním/odebráním možnosti udržení volantu ve stejné pozici
\end{itemize}

\podsekce{Fitness funkce}
Fitness funkce je stejně jako definice prostředí a agenta nesmírně důležitá. Nicméně nelze jí předem navrhnout
%- na vstup site privest i aktualni rychlost a natoceni volantu (to co
%ridic prece v kabine vidi, kdyz auto ridi)
%- mozna bych zrusil jednotlive rychlostni stupne, nechal bych jen
%zpatecku/brzdu a jizdu dopredu/akceleraci (tzn. jeden neuron je
%akcelerace a druhy brzda nebo couvani)


\kapitola{Implementace}
Vlastní práce se skládá ze dvou částí. Klientská část která slouží k~vizualizaci algoritmu a zobrazení výsledků ze serverové části. Serverová část pro maximální urychlení simulace. 

\sekce{Simulace}
Simulace je realizovaná jako knihovna pro Node.js, lze jí tedy použít jak u klientské částí, tak u serverové části. Poskytuje kompletní fyzikální simulaci agenta, prostředí ve kterém se pohybuje, jeho ovládání a výpočet fitness funkce. Součástí simulačního prostředí je také kód pro její vizualizaci.

Rychlost byla zajištěna implementací profilovacího programu (\textbf{benchmark.js} ve složce simulation), který spouští simulaci na předem připravené populaci jedinců. Výstupem je pak doba, za jakou jí vyhodnotil na jednom jádře procesoru. Tento údaj byl pak používán při implementaci simulace pro orientační představu, jak moc případné změny v kódu ovlivňují rychlost samotné simulace. Dále byla simulace podrobena občasnému profilování v klientské částí pomocí vývojářských nástrojů prohlížeče chrome, na kterém simulace jede nejlépe.

Nenáročnost, která souvisí s rychlostí pak byla zajištěna tím, že bylo v průběhu psaní kódu dbáno na to, aby v průběhu simulace nedocházelo k přebytečným alokacím, které by nejen že mohli způsobit přebytečný nárůst požadované paměti ale způsobovali by také nepředvídatelné zpomalení, které sebou přináší jazyk využívající garbage kolektor.

Robustnost je podrobněji vysvětlená v sekci \ref{sec:fitness} a popis toho, jak bylo dosaženo stejných podmínek pro všechny agenty lze nalézt v návrhu \ref{sec:ECS} především v popisu RoadManageru.

\podsekce{Fitness funkce}
\label{sec:fitness}
Fitness funkce je důležitou součástí simulace, která zásadně ovlivňuje chování výsledných agentů a je tedy nutné jí volit vhodně. Je nutné, aby funkce agenta motivovala ke správné činnosti.

Po několika pokusech a konzultaci s vedoucím práce byla jako metrika úspěchu agenta zvolena celková vzdálenost, kterou je agent schopný překonat v průběhu jedné generace. Výpočet je realizován pomocí RoadDirectoru, který si při každém přechodu zaznamená bod, ve kterém se po přesunu agent nachází. Výsledná fitness je pak součet uražených vzdáleností pro každou místnost. Road direktor si pro každou obrazovku uchovává vzdálenost, kterou agent v dané obrazovce překonal. Výsledným fitness je pak součet všech vzdáleností na všech obrazovkách.

\podsekce{Simulační prostředí}
\label{sec:simulationEnvironment}
Simulační prostředí poskytuje sadu překážek, kterou agent musí překonat. Testuje se tak, co je vlastně agent schopný se naučit. V zadání práce lze nalézt podmínku pro postupné stupňování obtížnosti, tohoto je dosaženo právě změnou simulovaného prostředí. V rámci těchto požadavků byly do road direktoru na-implementovány následující dílky.

Z obrázku \ref{fig:benchmarkcluster} a tabulky \ref{tbl:benchmark} lze vyčíst, že vyhodnocení 1000 jedinců o 1000 generací trvá něco okolo 9 h čistého výpočetního času a to do měření není započítáno to, že se agenti postupně zlepšují, což koreluje s delší dobou evaluace. 

Z tohoto důvodu bylo ověřování rozděleno do 3 částí různých obtížností. Volba obtížnosti pak probíhá volbou vhodného dílků (prostředí), který má nastavené patřičné návaznosti. Protože se dílek I objevuje v simulačním prostředí dvakrát je pro jednoduchost duplikován a jsou mu změněny patřičné navazující dílky dle obtížnosti.

První konfigurace probíhá tak, že je agent postaven doprostřed silnice ve tvaru I (obrázek \ref{fig:i}). Cílem je zjistit, zda je schopný se agent naučit jezdit (nebo alespoň couvat) rovně. V této konfiguraci je jedinou návazností (z obou možných směrů) samotný dílek I. Vzniká tak nekonečný tunel, kterým může agent cestovat.

\begin{figure}[H]
	\centering
	\includegraphics[scale=0.5]{pieces/I}
	\caption{Silnice ve tvaru písmene I}
	\label{fig:i}
\end{figure}
  

Další stupeň testuje schopnosti agenta vyhýbat se překážkám. Agent je opět postaven do silnice ve tvaru I ale tentokrát jsou v ní zdi, kterým se musí vyhnout (obrázek \ref{fig:iwithobstructions}). Návaznost z obou stran je opět samotný dílek (je to obdobně jako je tomu v první konfiguraci).

\begin{figure}[H]
	\centering
	\includegraphics[scale=0.6]{pieces/I_with_obstructions}
	\caption{Silnice ve tvaru I s překážkami}
	\label{fig:iwithobstructions}
\end{figure}

Poslední Konfigurace se skládá z více dílků a testuje, zda a jak moc je schopný se agent naučit otáčet. K tomuto mu slouží malá dráha ve tvaru O skládající se z níže zobrazených dílků. Agent stejně jako v prvním případě začíná v silnici ve tvaru I s tím, že zde probíhá výše zmíněná změna návazností a to při průjezdu horní částí obrazovky je zaměněn na obrázek \ref{fig:upsidel}. V případě, že se agent vydá dolu je přesměrován na silnici ve tvaru L je tak nucen po zvládnutí jízdy dopředu naučit se zatáčet.

\begin{figure}[H]
	\centering
	\includegraphics[scale=0.6]{pieces/upside_L}
	\caption{Silnice ve tvaru horizontálně obráceného L}
	\label{fig:upsidel}
\end{figure}

\begin{figure}[H]
	\centering
	\includegraphics[scale=0.6]{pieces/L}
	\caption{Silnice ve tvaru L}
	\label{fig:l}
\end{figure}
V případě, že odbočí doprava je přesměrován na níže uvedený dílek ve tvaru -. Toto ověří, zda se agent dokáže přeorientovat na horizontální pohyb.
\begin{figure}[H]
	\centering
	\includegraphics[scale=0.6]{pieces/-}
	\caption{Silnice ve tvaru -}
	\label{fig:-}
\end{figure}
Po projetí dílku ve tvaru I je přesměrován na jeden z následujících dílků.
\begin{figure}[H]
	\centering
	\includegraphics[scale=0.6]{pieces/inverted_L}
	\caption{Silnice ve tvaru zrcadlově obráceného L}
	\label{fig:invertedl}
\end{figure}
\begin{figure}[H]
	\centering
	\includegraphics[scale=0.6]{pieces/inverted_upside_L}
	\caption{Horizontálně a vertikálně obrácené L}
	\label{fig:invertedupsidel}
\end{figure}



\sekce{Serverová část}
Jak je jíž zmíněno v návrhu serverová část vyhodnocuje jednotlivé jedince distribuovaně s~pomocí fronty úkolů. Frontu poskytuje knihovna \textbf{Bull}, která používá \textbf{Redis} pro správu údajů o~jednotlivých úkolech. 

Cílem byla implementace robustního systému, který v~ideálním případě rozloží výpočetní zátěž mezi jednotlivé uzly rovnoměrně. Dalším požadavkem byla možnost odpojení kdykoliv kteréhokoliv z počítačů, jelikož ne všechny počítače je možné mít puštěné po celou dobu vyhodnocování.

Samotná implementace je dle návrhu rozdělena do dvou částí a to klientské části (implementace ve složce client), která obaluje algoritmus NEAT a posílá genomy na serverovou částí. Serverová část obaluje simulaci a vyhodnocuje genomy na základě jejích konfigurace.

\podsekce{Konfigurovatelnost}
Experimenty, které bude klientská část provádět jsou nastavitelné s pomocí dvou konfiguračních souborů.
Globální konfigurační soubor \emph{config.env}, který nastavuje proměnné prostředí umožňuje nastavit jedinou proměnnou a to FPS, která řídí jednotný krok simulačního enginu. Fixní krok je tu proto, aby se zajistilo, že všechny simulace běží ve stejných podmínkách.

Druhým jíž zajímavějším konfiguračním souborem je \emph{batch.json}. Tento soubor obsahuje informace o všech experimentech, které se mají spustit. Data o každém experimentu jsou uchována v jednoduchém JSON objektu, který obsahuje následující informace:

\begin{itemize}
	\item Nastavení knihovny Neataptic:
	\begin{itemize}
		\item INPUTS - Počet neuronů ve vstupní vrstvě ,
		\item OUTPUTS - Počet neuronů ve výstupní vrstvě
		\item POPSIZE - Velikost populace
		\item MUTATION\_RATE - Pravděpodobnost mutace
		\item ELITISM - Zkopíruje beze změny N nejlepších jedinců
		\item EQUAL - 
	\end{itemize}
	\item Nastavení simulace
	\begin{itemize}
		\item STARTING\_PIECE - Dílek na kterém agent začíná. Jejích popis lze nalézt v RoadManageru
		\item OPTIONS - Umožňuje zapnout některé z rozšíření (popsaných v kapitole \ref{sec:extensions}). Následující možnosti
		\begin{itemize}
			\item holdWheel - Možnost držení volantu ve fixní pozici
			\item inputVelocity - Přidává na vstup neuronové sítě aktuální rychlost
			\item inputWheel - Přidává na vstup neuronové sítě aktuální náklon volantu
		\end{itemize}
		\item TIME - Počet vteřin, který určuje maximální dobu simulace
	\end{itemize}
\end{itemize}

\podsekce{Výpočetní cluster}
\label{sec:cluster}
Ukázalo se, že vyhodnocování simulace zabírá neúměrné množství času a to i na nejvýkonnějším dostupném počítači. 
Například vyhodnocení jedné generace populace o~1024 jedincích zabralo ~290 s~na nejsilnějším dostupném pc. Z~tohoto důvodu bylo rozhodnuto o~distribuci výpočetní zátěže mezi více počítačů. Byl vytvořen výpočetní cluster se specifikací popsanou v~tabulce \ref{table:hw_table}.
\begin{table}[h!]
	\centering
	\begin{tabular}{|l|c|c|c|}
		\hline 
		Procesor & RAM & Počet & Architektura\\ 
		\hline 
		S5P6818 Octa core & 1 GB & 2 & arm64 \\ 
		\hline 
		Broadcom BCM2837B0 quad-core & 1 GB & 1 & arm32 \\ 
		\hline 
		Phenom X4 965 & 8 GB & 1 & x64 \\ 
		\hline
		Intel Core i5-2300 & 4 GB & 1 & x64 \\ 
		\hline 
		Intel atom x5-Z8350 & 2 GB & 1 & x64 \\ 
		\hline
		Cortex-A5 & 1 GB & 1 & armv7l \\
		\hline
	\end{tabular} 
	\caption{Použitý hardware}
	\label{table:hw_table}
	
\end{table}

Lze i namítnout, že se zde projevuje určitá režie při síťové komunikaci se serverem, což může být zdrojem určitého zpomalení.

Pro ověření rychlosti bylo provedeno měření výkonu clusteru a jeho porovnání s nejvýkonnější dostupnou sestavou. Měření bylo provedeno nad náhodně vygenerovanými populacemi. Jelikož se simulace může ukončit předčasně (například při kolizi s překážkou), byla simulace provedena pro každou velikost $10\times$ a výsledek byl zprůměrován. Naměřená data lze nalézt v tabulce \ref{tbl:benchmark} ze které vychází obrázek \ref{fig:benchmarkcluster} na kterém lze vidět výsledky tohoto srovnání. 

Porovnání rychlosti clusteru s nejvýkonnějším dostupným počítačem ukazuje, že cluster je ve většině případů skoro stejně nebo  výrazně rychlejší než samostatný výpočet. Jediné dvě naměřené instance, kde toto neplatí je u populací o velikosti 100 a 200, kde si cluster vede mírně hůře, než nejvýkonnější dostupná sestava. 
Toto lze vysvětlit jak přítomností méně výkonného hardwaru v clusteru (především se jedná o S5P6818) na které se musí u menších populací čekat. Pro ověření této teorie byl cluster spuštěn v dalších konfiguracích, kde byly postupně odebírány jednotlivé počítače a měření bylo opakováno. 

\begin{itemize}
	\item Cluster - Celý cluster
	\item Cluster-2 - Odebrán S5P6818 (obě jednotky)
	\item Cluster-3 - Odebrán Intel atom x5-Z8350
	\item Cluster-4 - Odebrán AMD A4-4300M
\end{itemize}
\begin{table}[H]
	\begin{tabular}{|l|c|c|c|c|c|}
		\hline
		Jedinců & Cluster & Cluster-2 & Cluster-3 & Cluster-4 & Phenom II X4 965 \\
		\hline
		100     & 9.853   & 6.1684    & 3.6887    & 3.7583    & 6.5186           \\
		\hline
		200     & 13.3993 & 9.126     & 6.451     & 7.1599    & 10.9839          \\
		\hline
		300     & 15.5628 & 10.5351   & 9.08      & 10.8206   & 16.8723          \\
		\hline
		400     & 17.7699 & 13.6263   & 11.9285   & 13.0231   & 23.5019          \\
		\hline
		500     & 19.1542 & 14.303    & 15.349    & 16.3707   & 31.7831          \\
		\hline
		600     & 20.4675 & 18.8677   & 18.571    & 20.0113   & 37.3242          \\
		\hline
		700     & 23.4671 & 20.1617   & 20.9039   & 23.7305   & 40.66            \\
		\hline
		800     & 25.07   & 23.3143   & 24.9978   & 26.3178   & 44.8019          \\
		\hline
		900     & 29.3611 & 28.1234   & 26.6288   & 29.6816   & 52.8829          \\
		\hline
		1000    & 30.5498 & 28.7635   & 29.5586   & 33.5195   & 62.8967          \\
		\hline
		1100    & 32.2825 & 32.0137   & 30.7372   & 37.2753   & 65.2363          \\
		\hline
		1200    & 34.3818 & 31.8152   & 34.0371   & 42.1325   & 72.2926          \\
		\hline
		1300    & 37.3422 & 35.8219   & 37.4416   & 40.3347   & 75.2937          \\
		\hline
		1400    & 38.4452 & 41.3896   & 38.6274   & 41.3493   & 81.1193          \\
		\hline
		1500    & 40.0487 & 43.225    & 43.4606   & 43.9233   & 81.3101 \\    
		\hline     
	\end{tabular}
	\caption{Naměřená data}
	\label{tbl:benchmark}
\end{table}
Je nutné však podotknout, že proměnlivá doba u vyhodnocování jedince znamená, že měření není zcela přesné. Nicméně lze na základě dat usoudit, že u větších populací dochází k přibližně $2\times$ zrychlení.

\begin{figure}[H]
	\centering
	\includegraphics[scale=0.7]{benchmarkCluster}
	\caption{Porovnání rychlosti clusteru s jedním PC}
	\label{fig:benchmarkcluster}
\end{figure}

\podsekce{Docker swarm}
Pro snadnou distribuci a správu byly všechny počítače zorganizovány do docker swarmu. Docker swarm obsahoval jednoho manažera (Broadcom BCM2837B0 quad-core), který zároveň spouštěl klientskou aplikaci a další služby:

Na manažeru nebyl spuštěn zpracovatel, aby se zabránilo jeho přetížení (manažer swarmu by měl být vždy dostupný).

Použití docker swarmu umožňuje především snadné nasazení a správu zpracovávajících procesů. Zároveň zajišťuje, že všechny instance zpracovatelů mají unifikovanou konfiguraci, což je zvláště důležité pro dosažení konzistentních výsledků.

\begin{figure}[H]
	\centering
	\includegraphics[scale=0.5]{distributed}
	\caption[Schéma distribuovaných výpočtů]{Schéma distribuovaných výpočtů}
	\label{fig:distributed}
\end{figure}

Tento přístup má několik výhod a to:

\begin{enumerate}
	\item Robustnost - Pokud jeden nebo více zpracovatelů selže (je například odpojen ze sítě) je možné pokračovat ve vyhodnocování (neúspěšný úkol lze vrátit zpátky do fronty). Toto v kombinaci s výše zmíněným docker swarmem znamená, že jakýkoliv výpočetní uzel lze kdykoliv vypnout a po znovu zapojení do sítě si načte nejnovější konfiguraci a začne znovu vyhodnocovat bez potřeby jakékoliv manipulace s jakoukoliv částí swarmu.
	\item Dobré rozložení zátěže - Jelikož si zpracovatel vytahuje úkoly z~fronty, je vždy optimálně zatížen, a není třeba řešit rozložení mezi různě výkonnými a zatíženými počítači.
	\item Škálovatelnost - problém lze škálovat až do doby, kdy počet procesorů nepřesáhne počet potřebných simulací. Chceme-li tedy vypočítat generaci o tisíci jedincích můžeme na ně nasadit až tisíc procesorů.
\end{enumerate}



\podsekce{Problémy implementace}
\label{sec:ImplementationTroubles}
Při implementaci serverové části a přehrávače byl objeven zásadní problém. Při vložení výsledného genomu do přehrávače se v některých případech rozcházela výsledná fitness agenta s fitness, která byla naměřena v přehrávači. 

Tuto odchylku lze vysvětlit nevyhnutelnou přítomností výpočtů pracující, které potřebují čísla s plovoucí desetinnou čárkou, kterou vyžaduje hlavně použitý fyzikální engine a výpočty, které probíhají při vyhodnocování výstupů z neuronové sítě a to z několika důvodů:

Ačkoliv je standard IEEE-754 deterministický nezaručuje stejný výsledek na rozdílném HW a už vůbec ne na HW v rozdílných architekturách. Problém může například nastat u volby zaokrouhlovacího módu, kterých standard IEEE-754 definuje několik a které mírně ovlivňují výslednou hodnotu.

Standard JavaScriptu nespecifikuje přesnou implementaci trigonometrických funkcí, které se hojně používají u vektorové algebry, která se opět využívá hlavně u výpočtů fyzikálního enginu.

I přes to, že odchylky způsobené výše uvedenými jevy jsou relativně malé je třeba myslet na to, že se akumulují s časem a také na to, že řídící neuronová síť se může za velmi obdobných podmínek zachovat diametrálně odlišně. 

\podsekce{Monitorování stavu}
Pro monitorování stavu algoritmu byla použita webová aplikace Grafana. Jak jíž bylo řečeno v sekci \ref{sec:grafana} jedná se o nástroj pro snadnou vizualizaci dat v databázi. V případě této práce posloužila k vytvoření jednoduchého kontrolního panelu, který obsahoval následující informace:
\begin{itemize}
	\item Graf zobrazující fitness nejlepšího/nejhoršího jedince dle generací
	\item Textové pole, které obsahuje nejlepší genom
	\item Numerické pole, které obsahuje počet generací
	\item Textové pole s konfigurací v jsonu
\end{itemize}
\begin{figure}[H]
	\centering
	\includegraphics[width=0.6\linewidth]{grafana}
	\caption{Kontrolní panel v aplikaci grafana}
	\label{fig:grafana}
\end{figure}

Serverová část má možnost vyhodnocovat více konfigurací zároveň, toto je reflektováno v návrhu kontrolního panelu, který umožňuje nejen zobrazovat a porovnávat jednotlivé konfigurace ale také dle nich filtrovat. Toto probíhá s pomocí menu v levém horním rohu.

\sekce{Uživatelská část}
Uživatelská část byla navržena tak, aby byla schopná vizualizovat průběh algoritmu NEAT a zároveň měla možnost znovu vyhodnocení existujících genomů vygenerovaných serverovou částí. První požadavek vznikl na základě konzultace s vedoucím, který chtěl algoritmus NEAT demonstrovat v hodinách předmětu VUI2. Druhý požadavek vznikl z důvodu potřeby vizualizace řešení, které generoval sever.

Klientská část je implementovaná jako \textbf{VUE.js} aplikace realizovaná s pomocí 3 rozdílných komponent. Kromě uvedených obrazovek se jedná o komponentu \emph{Simulation}, která obaluje samotnou simulaci a vykreslování agenta. Používá jí jak přehrávač, tak vizualizace v realném čase.
\podsekce{Vizualizace}
Vizualizace se skládá z jednoduchého rozhraní, které lze vidět na obrázku \ref{fig:visualization}. V horní části je graf, zobrazující průběh genetického algoritmu. Lze v něm nalézt fitness nejlepšího, nejhoršího a průměrného jedince v populaci.

Další část se skládá z konfigurovatelného množství simulačních prostředí. Jednotliví jedinci v generaci jsou pak rovnoměrně rozloženi mezi všechna simulační prostředí a uživatel může pozorovat vývoj jedinců v reálném čase.

Poslední tlačítko slouží k urychlení simulace. Způsobí to, že se simulace začne obnovovat bez vykreslování. Toto jí značně zrychlí.

\begin{figure}[h!]
	\centering
	\includegraphics[width=0.6\linewidth]{visualization}
	\caption{Uživatelské rozhraní klientské části}
	\label{fig:visualization}
\end{figure}

\podsekce{Přehrávač}
Přehrávač genomů, také poskytuje jednoduché rozhraní skládající se z textové částí pro vložení genomu (výstup ze serverové části lze zobrazit například v grafaně), graf pro zobrazení průběhu fitness agenta a vykreslovací část, která zobrazuje, jak si agent vedl. Pro menší výpočetní nároky je do grafu zapisována fitness každou vteřinu.  Výsledná komponenta se nachází na obrázku \ref{fig:player}.\\
Díky problémům s implementací naznačených v sekci \ref{sec:ImplementationTroubles} bylo třeba vymyslet alternativní způsob přehrávání. Nakonec byl návrh pozměněn tak, že si serverová část ukládá pro každý snímek pozice, úhel, aktuální fitness a dílek na kterém se agent nachází, kterou pak přehrávač interpretuje oproti klasickému přístupu, který by zahrnoval opětovnou simulaci agenta se stejným genomem. Tímto je dosaženo toho, že si lze zobrazit průběh agenta tak, jak tomu bylo při jeho vyhodnocování.

\begin{figure}[H]
	\centering
	\includegraphics[width=0.4\linewidth]{player}
	\caption{Rozhraní přehrávače}
	\label{fig:player}
\end{figure}


\sekce{Nasazení serverové části}
Jak jíž bylo zmíněno serverová část je obalená do docker kontejneru jednak pro snadné nasazení a jednak z důvodu snadného nasazení do swarmu, tak pro jednotné prostředí.

Nejdříve je potřeba vytvořit swarm. Tohoto lze dosáhnout zadáním příkazu \emph{docker swarm init} na počítači, který chceme používat jako swarm manager (lze také použít docker-machines a výpočty dělat na jednom počítači). Po jeho zadání se vytvoří docker swarm a zároveň je vypsán příkaz, který lze použít pro připojení dalších uzlů do swarmu (lze ho zobrazit i později s použitím příkazu \emph{docker swarm join-token worker}).

Nasazení probíhá nejdříve spuštěním příkazu \emph{docker-compose up} ve složce \textbf{client}. Toto kromě samotného klienta spustí jak následující služby:
\begin{itemize}
	\item Databáze
	\begin{itemize}
		\item Redis - pro \textbf{Bull}
		\item Postgresql - pro ukládání informací o průběhu algoritmu
	\end{itemize}
	\item Služby pro monitorování
	\begin{itemize}
		\item Grafana (port 3000) - pro monitorování stavu algoritmu NEAT
		\item Arena (port 4567) - pro monitorování stavu \textbf{Bull}
		\item Portainer (port 9000) - pro monitorování swarmu
		\item Adminer (port 8080) - pro správu databáze
	\end{itemize}
\end{itemize}
Po úspěšném provedení tohoto příkazu je třeba spustit zpracovatele na jednotlivých uzlech swarmu. Lze toho dosáhnout přechodem do složky \textbf{Server} a spuštěním příkazu \emph{docker deploy Server --compose-file docker-compose.yml}. Protože je tento příkaz v době psaní práce označená jako experimentální je třeba ho povolit. Tohoto je dosaženo modifikací souboru \textbf{daemon.json} (na linuxu je ho nutné vytvořit v \emph{/etc/docker/daemon.json}) přidáním řádku \emph{"experimental": true}.

Po tomto kroku je třeba ještě nastavit správnou ip adresu a port \textbf{Redis} serveru (ip adresa swarm mangeru) v souboru \emph{config.env}.

Po provedení těchto kroků je na každém uzlu spuštěn vyhodnocovatel, který začne automaticky stahovat a vyhodnocovat jednotlivé jedince. V případě vypnutí/pádu uzlu je úkol po čase vrácen do fronty a vyhodnocení probíhá na jiném uzlu.

Pokud se uživatel rozhodne používat portainer musí se z bezpečnostních důvodů přihlásit do 5 minut od jeho spuštění.

\sekce{Nasazení uživatelské části}
Uživatelskou část lze jednoduše nasadit s pomocí manageru závislostí \emph{npm}. Stačí ve složce s implementací spustit příkaz \emph{npm run dev} a pak si v prohlížeči otevřít adresu \emph{localhost:8080}. V případě přehrávače je to \emph{localhost:8080/\#/player}

\kapitola{Experimenty}
Po návrhu simulačního prostředí byl agent vyzkoušen v~několika situacích se stupňující se obtížností. Každá simulace probíhala s~100 jedinci po 500 generací. Doba vyhodnocování byla nastavena na 60 vteřin. Ačkoliv je pravděpodobné, že by delší doba evaluace by pravděpodobně vyústila v lepší výsledky její výpočet v různých konfiguracích se ukázal jako příliš časově náročný navíc empirické pozorování ukázalo, že tato konfigurace poskytuje dostatečně dobré výsledky za snesitelný čas. 

S ohledem na časovou náročnost výpočtů byly zkoušeny jen konfigurace popsané v návrhu (sekce \ref{sec:extensions}) na implementovaných dílcích k náhledu v sekci \ref{sec:simulationEnvironment}.

V rámci časové úspory je každý stupeň testován jen 2x a to bez jakýchkoliv rozšíření (vstupů/výstupů) a poté se všemi dostupnými. Cílem je zjistit, jak moc tato rozšíření ovlivňují výsledného agenta.

\sekce{První stupeň - silnice ve tvaru I}
%Silnice ve tvaru I byla v obou případech agentem úspěšně překonána a to již za 2 generace. Je zajímavé, že ačkoliv byl výsledný pohyb agenta stejný (jízda dozadu relativně rovně) jsou zde odlišnosti ve výsledných neuronových sítí agenta v obou konfiguracích.
\podsekce{Konfigurace bez rozšíření}
%Konfigurace bez rozšíření dosáhla rovné jízdy (kývavým pohybem) již v druhé generaci a postupně jen zlepšovala míra oscilace.
%\begin{figure}[H]
%	\centering
%	\includegraphics[width=1.0\linewidth]{solutions/Ibasic/basicGraph}
%	\caption{Průběh evaluace}
%	\label{fig:basicgraph}
%\end{figure}
%Při pohledu na výslednou neuronovou síť, která 
%\begin{figure}[H]
%	\centering
%	\includegraphics[width=0.6\linewidth]{solutions/Ibasic/basic}
%	\caption{Výsledná neuronová síť}
%	\label{fig:basic}
%\end{figure}


\podsekce{Konfigurace s rozšířeními}
%\begin{figure}[H]
%	\centering
%	\includegraphics[width=1.0\linewidth]{solutions/Ibasic/advancedGraph}
%	\caption{Průběh evaluace}
%	\label{fig:basicgraph}
%\end{figure}
%\begin{figure}[H]
%	\centering
%	\includegraphics[width=0.6\linewidth]{solutions/Ibasic/advanced}
%	\caption{Výsledná neuronová síť}
%	\label{fig:basic}
%\end{figure}
\sekce{Druhý stupeň - silnice ve tvaru I s překážkami}
\podsekce{Konfigurace bez rozšíření}
\podsekce{Konfigurace s rozšířeními}

\sekce{Třetí stupeň - Okruh}
\podsekce{Konfigurace bez rozšíření}
\podsekce{Konfigurace s rozšíření}

\kapitola{Možná vylepšení}
Při implementaci aktuálního řešení bylo zjištěno, že řešený problém je mnohem výpočetně náročnější než se při původním návrhu čekalo. Toto vedlo k navrhu distribuovaného řešení, které ovšem se sebou nese problémy v rozdílech u výpočtů s desetinou čárkou.

Z tohoto důvodu by experimentu prospělo použití homogenního clusteru. Předešlo by se problémům, který se sebou přináší gridový cluster.

Další možným rozšířením by bylo spuštění řešení na clusteru s větším výpočetním výkonem. Toto by umožnilo vyzkoušet více experimentů a konfigurací, což by vedlo k lepší představě o  možnostech samotného algoritmu. Navíc by to umožnilo plně využít potenciál, který nabízí oddělený simulační kód (simulace s dynamickými prvky, ...). 

Poslední možností by byla reimplementace existujícího řešení v některém ze systémových jazyků (\textbf{C}, \textbf{C++}, \textbf{Rust}). Došlo by tak k určitému zrychlení výpočtů a bylo by tedy možné vyhodnotit více jedinců na jednom počítači než doposud. 

Grafickou část aplikace by bylo možné rozšířit o různá nastavení, která jsou v tuto chvíli dostupná jen ve zdrojovém kódu aplikace.

\kapitola{Závěr}
Tato práce se zabývala návrhem, tvorbou prostředí pro simulovaného agenta a jeho řízení s pomocí upraveného algoritmu neuroevoluce. Následoval návrh, provedení a zhodnocení experimentů, které byly navrženy, tak aby testovaly možnosti agenta.


%\input{zaverprace.tex}
\begin{literatura}
	\citace{practitionersApproach}{BUDUMA, Nikhil 2017}{
		\autor{BUDUMA, Nikhil.} \nazev{Fundamentals of deep learning: designing next-generation machine intelligence algorithms.} Sebastopol: O'Reilly, 2017. ISBN 978-149-1925-614.}
	\citace{fundementalsOfDeepLearning}{PATTERSON, Josh. 2017}{\autor{PATTERSON, Josh.} \nazev{Deep learning : a practitioner's approach}
		\nazev{Deep learning : a practitioner's approach. 1.} Beijing ; Boston ; Farnham ; Sebastopol ; Tokyo: O'Reilly, 2017. ISBN 978-1-491-91425-0.}
	\citace{geneticAlgorithms}{MITCHELL, Melanie., 1996}{\autor{MITCHELL, Melanie.} \nazev{An introduction to genetic algorithms.} Cambridge: Bradford Book, c1996. ISBN 0-262-13316-4.}
	\citace{geneticCZ}{HYNEK, Josef., 2008}{\autor{HYNEK, Josef.} \nazev{Genetické algoritmy a genetické programování.} Praha: Grada, 2008. Průvodce (Grada). ISBN 978-80-247-2695-3.}
	\citace{differentialEvolution}{LÝSEK Jiří, ŠŤASTNÝ Jiří, 2014}{
	\autor{LÝSEK, Jiří} a \autor{ŠŤASTNÝ, Jiri.} (2014). \nazev{Automatic discovery of the regression model by the means of grammatical and differential evolution.} Agricultural Economics (AGRICECON). 60. 546-552. 10.17221/160/2014-AGRICECON. }
	\citace{NEAT}{STANLEY, Kenneth O, Risto MIIKKULAINEN., 2002}{\autor{STANLEY, Kenneth O.} a \autor{Risto MIIKKULAINEN.} \nazev{Evolving Neural Networks through Augmenting Topologies.} In: Evolutionary Computation [online]. 2002, 10(2), s. 99-127 [cit. 2018-12-08]. DOI: 10.1162/106365602320169811. ISSN 1063-6560. Dostupné z: http://www.mitpressjournals.org/doi/10.1162/106365602320169811}
	
	\citace{odNEAT}{\autor{SILVA, Fernando}, \autor{Paulo URBANO}, \autor{Luís CORREIA} a \autor{Anders Lyhne CHRISTENSEN}, 2015}{\autor{SILVA, Fernando}, \autor{Paulo URBANO}, \autor{Luís CORREIA} a \autor{Anders Lyhne CHRISTENSEN}. \nazev{OdNEAT: An Algorithm for Decentralised Online Evolution of Robotic Controllers. } Evolutionary Computation. 2015, 23(3), 421-449. DOI: 10.1162/EVCO\_a\_00141. ISSN 1063-6560. Dostupné také z: \url{http://www.mitpressjournals.org/doi/10.1162EVCO_a_00141}}
	\citace{hyperNEAT}{\autor{STANLEY, Kenneth O.}, \autor{David B. D'AMBROSIO} a \autor{Jason GAUCI.}, 2009}{\autor{STANLEY, Kenneth O.}, \autor{David B. D'AMBROSIO} a \autor{Jason GAUCI.} \nazev{A Hypercube-Based Encoding for Evolving Large-Scale Neural Networks. Artificial Life [online].} 2009, 15(2), 185-212 [cit. 2018-12-15]. DOI: 10.1162/artl.2009.15.2.15202. ISSN 1064-5462. Dostupné z: \url{http://www.mitpressjournals.org/doi/10.1162/artl.2009.15.2.15202}}
	\citace{aiModernApproach}{\autor{RUSSELL, Stuart J.}, \autor{Peter NORVIG} a \autor{Ernest DAVIS}, 2010}{\autor{RUSSELL, Stuart J.}, \autor{Peter NORVIG} a \autor{Ernest DAVIS}. \nazev{Artificial intelligence: a modern approach. 3rd ed.} Boston: Pearson, c2010. Prentice Hall series in artificial intelligence. ISBN 978-0-13-207148-2.}
	\citace{kozaGP}{\autor{KOZA, John R.}, 1992}{\autor{KOZA, John R.} \nazev{Genetic programming: on the programming of computers by means of natural selection.} Cambridge, Mass.: MIT Press, c1992. ISBN 0-262-11170-5.}
	\citace{VUE}{\autor{MACRAE, Callum.}, 2018}{\autor{MACRAE, Callum.} \nazev{Vue.js: up and running: building accessible and performant web apps.} Sebastopol, California: O'Reilly Media, [2018]. ISBN 1491997249.}
	\citace{UBER}{\autor{PETROSKI SUCH, Felipe} a kol., 2018}{\autor{PETROSKI SUCH, Felipe}, \autor{Vashisht MADHAVAN}, \autor{Edoardo CONTI}, \autor{Joel LEHMAN}, \autor{Kenneth O. STANLEY} a \autor{Jeff CLUNE}. \nazev{Deep Neuroevolution: Genetic Algorithms Are a Competitive Alternative for Training Deep Neural Networks for Reinforcement Learning} [online]. [cit. 2019-01-01]. DOI: arXiv:1712.06567v1. Dostupné z: https://arxiv.org/abs/1712.06567v1}
	\citace{VUE}{Vue.js, 2014}{\nazev{Vue.js}. Vue.js [online]. Copyright © 2014 [cit. 01.01.2019]. Dostupné z: \url{https://vuejs.org/}}
	\citace{Curiosity}{\autor{STANTON, Christopher}, \autor{Jeff CLUNE} a a \autor{Josh BONGARD}, 2016}{STANTON, Christopher, Jeff CLUNE a Josh BONGARD. Curiosity Search: Producing Generalists by Encouraging Individuals to Continually Explore and Acquire Skills throughout Their Lifetime. PLOS ONE [online]. 2016, 11(9) [cit. 2019-01-02]. DOI: 10.1371/journal.pone.0162235. ISSN 1932-6203. Dostupné z: http://dx.plos.org/10.1371/journal.pone.0162235}
	\citace{CuriosityFitness}{\autor{SCHAUL, Tom}, \autor{Yi SUN}, \autor{Daan WIERSTRA}, \autor{Fausino GOMEZ} a \autor{Jurgen SCHMIDHUBER}, 2018}{\autor{SCHAUL, Tom}, \autor{Yi SUN}, \autor{Daan WIERSTRA}, \autor{Fausino GOMEZ} a \autor{Jurgen SCHMIDHUBER}. \nazev{Curiosity-driven optimization}. In: 2011 IEEE Congress of Evolutionary Computation (CEC) [online]. IEEE, 2011, 2011, s. 1343-1349 [cit. 2019-01-02]. DOI: 10.1109/CEC.2011.5949772. ISBN 978-1-4244-7834-7. Dostupné z: \url{http://ieeexplore.ieee.org/document/5949772/}}
\end{literatura}

\prilohy{}
\priloha{CD se zdrojovým kódem}

\end{document}

