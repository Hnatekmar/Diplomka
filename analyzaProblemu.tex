\kapitola{Analýza problému}
Tato kapitola se zabývá analýzou funkčních a nefunkčních požadavků pro softwarové řešení. Požadavky vznikaly na základě konzultace s vedoucím a vlastní invencí.

\sekce{Funkční požadavky}
Funkční požadavky jsou rozděleny do několika kategorii.

\podsekce{Simulace}
Vyhodnocování agenta bude probíhat jeho nasazením v simulovaném prostředí. Fitness bude pak vyhodnocena na základě jeho akcí v prostředí.
\begin{itemize}
	\item Testovací prostředí musí být pro všechny agenty stejné
	\item Fitness funkce musí být deterministická
	\item Fitness by měla být zjistitelná kdykoliv v průběhu simulace
	\item Simulace by měla být konfigurovatelná 
	\item Možnost změny obtížnosti simulace pro agenta
	\item Měla by existovat možnost spuštění více instancí simulace v rámci jednoho programu
\end{itemize}
\podsekce{Vizualizace}
Průběh algoritmu je třeba zobrazit.
\begin{itemize}
	\item Je třeba provést grafickou vizualizaci fyzikální simulace
	\item Při zobrazení by mělo být možné vyčíst stav agenta (fitness, senzory, ...) 
	\item Možnost vizualizace průběhu simulace v reálném čase (například pro její demonstraci v předmětu VUI2)
\end{itemize}

\sekce{Nefunkční požadavky} 
Spolu s funkčními požadavky jsou na řešení kladeny také požadavky nefunkční.
\begin{itemize}
	\item Škálovatelnost - Možnost spustit a vykreslit libovolné množství simulací
	\item Rychlost simulace. Simulace by měla být schopná samostatně běžet alespoň rychlostí 30 snímků za vteřinu.
	\item Robustnost - Simulace by měla být odolná neočekávaným situacím
	\item Portabilita - bylo třeba zajistit, aby šlo kód rozběhnout v různých platformách v různých konfiguracích.
	\item Robustnost - Simulace by si měla poradit s neočekávanými vstupy, jako je třeba \textbf{NaN}, který vychází z neuronové sítě
\end{itemize}

